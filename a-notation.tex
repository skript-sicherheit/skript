\chapter{Glossar}
\section{Mathematische Bezeichnungen}

\setlength{\tabcolsep}{12pt}
\def\arraystretch{1.5}%  1 is the default, change whatever you need
\begin{tabular}{l|l}
	$\mathbbm{Z}^{\ast}_p$ & Zyklische multiplikative Gruppe ganzer Zahlen, die kleiner $p$ und koprim zu $p$ sind, das heißt $\{x : \ggT(x, p) = 1\}$\\
	$\mathbbm{Z}_N$ & Zyklische additive Gruppe ganzer Zahlen modulo $N$, das heißt $\{0, \dots, N-1\}$\\
	$\mathbbm{F}^{\ast}_q$ & Multiplikative Gruppe des dazugehörigen Galoiskörpers $\mathbbm{F}_q$
\end{tabular}

\section{Notationsformalismus}
\begin{tabular}{l|l}
	$A \vert B$ & kdsaldjskladjalksj\\
	$\A^{\B}$ &dsa\\
	$Adv^{cr}_{H,\A}(k)$ & dsa \\
	$Adv^{ow}_{H,\A}(k)$ & dsa \\
	$Adv^{tcr}_{H,\A}(k)$ & dsa
\end{tabular}

\section{Komplexitätsklassen}
\begin{tabular}{l|l}
	$P$ & dsa \\
	$NP$ & dsa
	%NP-vollständig
\end{tabular}

\section{Begriffserklärungen}

%Prüfsumme
%Padding
%Kollision
%Urbildraum / Bildraum
%gleichverteilt
%Teta / Groß-O, Omega
%Gruppe (additiv, multiplikativ)
%Untergruppe
%Gruppenordnung
%Inverses Element (Multiplikativ)
%Semantik
%deterministisch
% a <- M, probabilistische Zuweisung einer Variablen 
%probabilistisch
%Homomorphie
%Heuristik
%diskreter Logarithmus
%Kryptosystem
%Permutation
%Unterscheider
%Schlüsselzentrale
%Replay-Attacke
%Sitzungsschlüssel
%Client / Server 
%forward-secrecy
%Meet-in-the-Middle
%Man-in-the-Middle
%kryptografische Hashfunktion
%binäre Suche
%Reduktionsfunktion