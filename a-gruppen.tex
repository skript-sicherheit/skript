
\chapter{Gruppen}
Gruppen sind algebraische Strukturen, die in vielen kryptographischen
Verfahren Anwendung finden. Sie werden in der Grundlagenvorlesung
\glqq Lineare Algebra\grqq~ausführlich behandelt. Hier sollen aber noch einmal
einige grundlegende Dinge wiederholt werden, die für die Kryptographie
und die Vorlesung \glqq Sicherheit\grqq~wichtig sind. 

Gruppen sind eine Abstraktion, mit der man unter anderem übliche
Rechenregeln verallgemeinern kann. Eine Gruppe besteht aus einer Menge
(z.B. von Zahlen) und einer Verknüpfung. Diese Verknüpfung bildet zwei
Elemente der Menge auf ein Mengenelement ab. Dass die Verknüpfung nur
auf Gruppenelemente abbildet, wird als \emph{Abgeschlossenheit}
bezeichnet.

Die Verknüpfung hat auf der Menge ein \emph{neutrales
  Element}. Verknüpft man ein Element $x$ aus der Gruppe mit diesem
Element, so erhält man wieder $x$. Bei einer Gruppe, die die Addition
als Verknüpfung hat, ist dies beispielsweise die $0$.

Für jedes Element $x$ enthält eine Gruppe ein \emph{inverses Element}
$x^{-1}$. Verknüpft man $x$ und $x^{-1}$, so erhält man das neutrale
Element. Für eine Gruppe mit Addition ist dies $-x$, bei einer Gruppe
mit Multiplikation $\frac{1}{x}$.

Außerdem ist die Verknüpfung \emph{assoziativ}. Das bedeutet, dass in
verschachtelten Ausdrücken die Reihenfolge, in der man die Verknüpfung
auf die Elemente anwendet, egal ist.

\begin{definition}[Gruppe]
Sei \G~eine Menge und $\cdot$ eine Verknüpfung mit $\cdot: \G\times \G \to
\G$. Dann heißt $(\G, \cdot)$ eine \textit{Gruppe}, wenn die Operation
in der Menge abgeschlossen ist (d.h. dass man durch Ausführen der
Operation auf Elementen der Menge keine Elemente erhält, die nicht in der
Menge liegen), es ein neutrales Element gibt, jedes 
Element ein Inverses hat und die Operation assoziativ ist.
\begin{itemize}
\item[Abgeschlossenheit:] \G~ist abgeschlossen bezüglich $\cdot$, also gilt:
  \[\forall x, y \in \G: (x\cdot y)\in \G\]
\item[neutrales Element:] Es existiert ein \textit{neutrales Element}  $e \in \G$ , sodass für
  alle $x \in \G$ gilt $x\cdot e=x=e\cdot x$.
\item[inverses Element:] Jedes Element aus \G~hat ein \textit{inverses Element}, d.h. es gilt $\forall x \in
  \G~ \exists y \in \G: x\cdot y=e=y\cdot x$.

\item[Asoziativität:] $\cdot$ ist assoziativ, d.h. es gilt $\forall x, y, z  \in ~\G:
  (x\cdot y)\cdot z = x\cdot(y \cdot z)$.
\end{itemize}
\end{definition}

Für die Gruppen, die in der Kryptographie Anwendung finden, gilt zudem,
dass die Verknüpfung kommutativ ist. Das bedeutet, dass die Reihenfolge
der Elemente, auf die Verknüpfung angewendet wird, egal ist.
\begin{definition}[kommuntative Gruppe]
  Eine Gruppe $(\G, \cdot)$ heißt \textit{kommutativ}, wenn gilt:
\[
  \forall x, y \in \G: x \cdot y = y \cdot x
\]
\end{definition}


\begin{definition}[Untergruppe]
  Sei $(\G, \cdot)$ eine Gruppe und $\mathbb{U} \subset \G$. Wenn
  $(\mathbb{U}, \cdot)$ wieder eine Gruppe ist, so nennt man $(\mathbb{U},
  \cdot)$ \textit{Untergruppe} von $(\G, \cdot)$.
\end{definition}


Als Beispiel wird geprüft, ob die natürlichen Zahlen mit der Addition
als Verknüpfung eine Gruppe bilden:
\begin{beispiel}
  Um zu prüfen, ob \((\Z{}, +)\) eine Gruppe ist, müssen
  wir nur die oben genannten Eigenschaften nachrechnen:
  \begin{itemize}
  \item Für die Abgeschlossenheit muss gelten: 
    \[\forall x, y \in \Z{}: (x+y)\in \Z{}. \] 
     Dies ist erfüllt. Die Summe zweier natürlicher Zahlen ist wieder
     eine natürliche Zahl.
  \item Es muss ein neutrales Element geben. Da gilt 
    \[\forall x \in \Z{}: x+0 = x = 0+x, \] ist $0$ das neutrale Element.
  \item Es muss für alle Elemente ein inverses geben. Dies ist gegeben
    durch \[\forall x \in \Z{}: x + (-x) = 0.\]
  \item Die Verknüpfung muss asoziativ sein. Es gilt:
    \[\forall x, y, z \in \Z{}: x + (y+z)=(x+y)+z,\] also ist die
    Addition asoziativ.
  \end{itemize}
  Damit ist gezeigt, dass $(\Z{}, +)$ eine Gruppe ist.
\end{beispiel}


Wird die Verknüpfung einer Gruppe als Multiplikation aufgefasst,
verwendet man häufig eine Exponentenschreibweise, um das mehrfache
Ausführen der Verknüpfung auszudrücken. Diese ist beispielsweise bei den
RSA- und ElGamal-Verschlüsselungs-Systemen verwendet:

\begin{definition}[Exponentenschreibweise für Gruppen]
 Dabei ist $g\in \G$, $m \in \Z{}$.
\[ g^m = \underbrace{g \cdot \dotsc \cdot g}_{\text{m mal}} \]
Hierbei halten die gewohnten Rechenregeln für Exponenten: $g^m\cdot g^n =
g^{m+n}$, $(g^m)^n = g^{mn}$
\end{definition}

\begin{definition}[Ordnung einer Gruppe]
  Sein $(\G, \cdot)$ eine Gruppe. Dann heißt die Anzahl $|\G|$ der Elemente in
  \G~die Ordnung der Gruppe.
\end{definition}

\begin{definition}[Ordnung eines Gruppenelements]
  Die Ordung $\operatorname{ord}(g) $ eines Gruppenelements $g$
  bezeichnet die kleinste 
  natürliche Zahl $n>0$, für die $g^n = e$ gilt. Existiert kein solches
  Element, so sagt man, $g$ habe unendliche Ordnung.
 \end{definition}

Für den Korrektheitsbeweiß des RSA-Verschlüsselungsverfahrens wurde der
kleine Satz von Fermat verwendet: 
 \begin{theorem}[Kleiner Satz von Fermat]
   Sei $(\G, \cdot)$ eine Gruppe, $a \in \Z{}$ und $p$ eine
   Primzahl. Dann gilt
   \[a^{p-1} \equiv 1 \mod p\]
 \end{theorem}
\section{Zyklische Gruppen}
Eine zyklische Gruppe hat die Eigenschaft, dass es einen Generator $g$
gibt. Das ist ein Gruppenelement, von dem aus man durch Potenzieren
jedes andere Gruppenelement erreichen kann. Wichtige Beispiele von
zyklischen Gruppen sind Restklassengruppen $(\Z{}/mZ, +)$ oder
elliptische Kurven.

\begin{definition}[Zyklische Gruppe]
  Sei $(\G, \cdot)$ eine Gruppe. Dann heißt $(\G, \cdot)$ zyklisch, wenn
  es ein $g \in \G$ gibt, sodass gilt
  \[
    \forall x \in \G: \exists n \in \Z{}: x=g^n,
  \]
  wobei gilt:
\[
 g^n = 
  \begin{cases} 
   \overbrace{g \cdot \dotsc \cdot g}^{\text{n mal}} & \text{falls } n > 0 \\
   g^e       & \text{falls } n = 0 \\
   \underbrace{g^{-1} \cdot \dotsc \cdot g^{-1}}_{\text{n mal}} &
   \text{falls }n < 0
  \end{cases}
\]

$g$ heißt \textit{Erzeuger} von $(\G, \cdot)$. 
\end{definition}
Jede zyklische Gruppe ist kommutativ.

Im Bereich der Sicherheit hat man es fast ausschließlich mit endlichen
Gruppen zu tun. Für endliche zyklische Gruppen reicht es, die Fälle mit
$n>0$ zu betrachten. 

\begin{theorem} 
Wenn die Ordung einer zyklischen Gruppe eine Primzahl ist, dann ist
jedes Element mit Ausnahme des neutralen Elements ein Generator. 
\end{theorem}

Es gibt auch endliche Gruppen, die nicht zyklisch sind (z.B. die
sog. \emph{Kleinsche Vierergruppe}). Eine Beispiel für eine unendliche
zyklische Gruppe ist $(\Z{}, + )$.
\section{Die Gruppe $\Z{N}^*$}
Es ist $\Z{N}$ die Menge der Restklassen der Multiplikation modulo $N$
(Umgangssprachlich kann man sich diese als die Menge der Zahlen
${0,\dots , N-1}$ vorstellen). Die Operation $\cdot$ wird als
Multiplikation modulo $N$ aufgefasst, also 
\[
a \cdot b = ab \mod N.
\]
 Offentsichtlich
ist $(\Z{N}, \cdot)$ keine Gruppe, denn $0$ hat kein inverses Element
bezüglich der Multiplikation.
Trotzdem gibt es Gruppen ganzer Zahlen bezüglich der
Multiplikation. Es reicht im Allgemeinen nicht aus, lediglich die $0$ zu
entfernen, damit $(\Z{N}\setminus \{0\}, \cdot)$ eine Gruppe ist. Sei
z.B. $N=6$ und damit $\Z{6}\setminus \{0\} = \{1,2,3,4,5\}$. Damit ist  
\[
2 \cdot 3 = 6 \equiv 0 \notin \Z{6}\setminus \{0\},
\]
also ist $\Z{6}\setminus \{0\}$ nicht abgeschlossen bezüglich $\cdot$. 
Dieses Problem lässt sich lösen, indem man alle weiteren Elemente $x$ aus
der Menge entfernt, für die $\ggt(x, N)\neq 1$ gilt. Wir bezeichnen eine
solche Gruppe mit $\Z{N}^*$. 

Für eine Primzahl $p$ gilt, dass  $|\Z{p}^*|= p-1$, denn für jede Zahl $x\in
{1,..., p-1}$ gilt, dass $\ggt(x, p)=1$. die Gruppe $\Z{p}^*$ enthält
also als Elemente die Äquivalenzklassen der Zahlen von $1$ bis $p-1$.

\begin{beispiel}
\label{bsp:azyklische_gruppe}
Betrachten wir die Gruppe $\G := (\Z{8}, \cdot)$. Zunächst prüfen wir, welche
Elemente die Gruppe hat:
\begin{eqnarray*}
ggT(1, 8) = 1 \rightarrow 1 \in \G\\
ggT(2, 8) = 2 \rightarrow 2 \notin \G\\
ggT(3, 8) = 1 \rightarrow 3 \in \G\\
ggT(4, 8) = 2 \rightarrow 4 \notin \G\\
ggT(5, 8) = 1 \rightarrow 5 \in \G\\
ggT(6, 8) = 2 \rightarrow 6 \notin \G\\
ggT(7, 8) = 1 \rightarrow 7 \in \G\\
\end{eqnarray*}
Wir prüfen nun, ob die Gruppe einen Generator hat. Für einen Generator
$g \in \G$ gilt 
\[
\forall x \in \G \exists n\in\{1,\dots, 7\}: x = g^n.
\]
Es kann $1$ kein Generator sein, denn $1^n=1$. Geprüft werden nun alle
weiteren Elemente:
\begin{eqnarray*}
  3^1 & = & 3\\
  3^2 & = & 9 \equiv 1 \Rightarrow 3^n\in\{1, 3\}\\
  5^1 & = & 5\\
  5^2 & = & 25 \equiv 1 \Rightarrow 5^n\in\{1, 5\}\\
  7^1 & = & 7\\
  7^2 & = & 49 \equiv 1 \Rightarrow 7^n\in\{1, 7\}\\
\end{eqnarray*}
Die Gruppe $\G$ hat aslo keinen Generator, ist also nicht zyklisch.
\end{beispiel}

\begin{beispiel}
  Analog zu Beispiel \ref{bsp:azyklische_gruppe} prüfen wir, ob $\G :=
  (\Z{7}, \cdot)$ zyklisch ist. $\G$ hat die Elemente $\{1, 2, 3, 4, 5,
  6\}$.

  $1$ ist das neutrale Element und damit kein Generator. Nun werden die
  anderen Elemente daraufhin geprüft, ob sie Generator sind:
  \begin{eqnarray*}
    2^1 &= & 2\\
    2^2 &= & 4\\
    2^3 &= & 8 \equiv 1 \Rightarrow 2^n \in \{1, 2, 4\}\\
\\    
    3^1 &= & 3\\
    3^2 &= & 9 \equiv 2\\
    3^3 &= & 27 \equiv 6\\
    3^4 &= & 81 \equiv 4\\
    3^5 &= & 243 \equiv 5\\
    3^6 &= & 729 \equiv 1 \Rightarrow \langle 3\rangle = \G
  \end{eqnarray*}
  Es ist also $3$ ein Generator. Damit ist die Gruppe zyklisch.

\end{beispiel}