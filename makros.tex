%%%%%%%%%%%%%%%%%%%%%%%%%%%%%%%%%%%%%%%%
%%% Makros für einheitliche Notation %%%
%%%%%%%%%%%%%%%%%%%%%%%%%%%%%%%%%%%%%%%%

\newcommand{\changefont}[3]{\fontfamily{#1} \fontseries{#2} \fontshape{#3} \selectfont}

%%% Notation für Gruppen, Ringe und Körper
\newcommand{\N}{\ensuremath{\mathbbm{N}}}
\newcommand{\No}{\ensuremath{\mathbbm{N}_0}}
\newcommand{\Z}[1]{\ensuremath{\mathbbm{Z}_{#1}}}
\newcommand{\Zx}[1]{\ensuremath{\mathbbm{Z}_{#1}^{\times}}}
\newcommand{\Q}{\ensuremath{\mathbbm{Q}}}
\newcommand{\R}{\ensuremath{\mathbbm{R}}}
\renewcommand{\P}{\ensuremath{\mathbbm{P}}}
\newcommand{\F}[1]{\ensuremath{\mathbbm{F}_{#1}}}
\newcommand{\Fx}[1]{\ensuremath{\mathbbm{F}_{#1}^{\times}}}
\newcommand{\K}{\ensuremath{\mathbbm{K}}}
\newcommand{\G}{\ensuremath{\mathbbm{G}}}
\newcommand{\Gx}{\ensuremath{\mathbbm{G}^{\times}}}

% For compatibility.
\newcommand{\calA}{\ensuremath{\mathcal{A}}}
\newcommand{\calC}{\ensuremath{\mathcal{C}}}
\newcommand{\calL}{\ensuremath{\mathcal{L}}}
\newcommand{\calO}{\ensuremath{\mathcal{O}}}
\newcommand{\calP}{\ensuremath{\mathcal{P}}}
\newcommand{\calR}{\ensuremath{\mathcal{R}}}
\newcommand{\calS}{\ensuremath{\mathcal{S}}}

%%% Notation für Kryptographie und Sonstiges
\newcommand{\plaint}{\ensuremath{M}}
\newcommand{\ciphert}{\ensuremath{C}}
\newcommand{\key}{\ensuremath{K}}
\newcommand{\skey}{\ensuremath{sk}}
\newcommand{\pkey}{\ensuremath{pk}}
\newcommand{\secpara}{\ensuremath{k}}
\newcommand{\enc}{\textsc{Enc}}
\newcommand{\dec}{\textsc{Dec}}
\newcommand{\sig}{\textsc{Sig}}
\newcommand{\ver}{\textsc{Ver}}
\newcommand{\keygen}{\textsc{KeyGen}}
\newcommand{\gen}{\textsc{Gen}}
\newcommand{\hash}{\ensuremath{h}}
\newcommand{\Com}{\textsc{Com}} %Commitment-Algorithmus
\newcommand{\A}{\ensuremath{\mathcal{A}}}
\newcommand{\advA}{\ensuremath{\mathcal{A}}}
\newcommand{\B}{\ensuremath{\mathcal{B}}}
\newcommand{\C}{\ensuremath{\mathcal{C}}}
\newcommand{\Sim}{\ensuremath{\mathcal{S}}} % Simulator
\newcommand{\ext}{\ensuremath{\mathcal{E}}} % Extraktor
\newcommand{\pw}{\ensuremath{\texttt{pw}}}

\newcommand\adv[2]{\mathbf{Adv}^{#1}_{#2}}
\DeclareMathOperator{\ggT}{ggT}
\DeclareMathOperator{\ggt}{ggT}
\newcommand*{\rArrow}{\ensuremath{\rightarrow}}
\newcommand*{\concat}{\ensuremath{\mathbin{\|}}} % Konkatenationssymbol %

\newcommand{\randUnif}{\xleftarrow{\textdollar}}

%Index-Macros

%Stromchiffren
\newcommand{\indexCaesarROT}{\index{ROT-13}}
\newcommand{\indexVignere}{\index{Vigenère-Chiffre}}
\newcommand{\indexCaesar}{\index{Caesar-Chiffre}}
\newcommand{\indexOTP}{\index{One-Time-Pad}}
\newcommand{\indexLFSR}{\index{Linear Feedback Shift Register (LFSR)}}

%Pseudozufall
\newcommand{\indexSeed}{\index{Seed|see {Pseudozufallszahlengenerator}}}
\newcommand{\indexPNS}{\index{Pseudozufallsfolge}}
\newcommand{\indexPRNG}{\index{Pseudozufallszahlengenerator}}

%Betriebsmodi
\newcommand{\indexECB}{\index{Betriebsmodus!Electronic Codebook Mode (ECB-Modus)}\index{Electronic Codebook Mode (ECB-Modus)}}
\newcommand{\indexCBC}{\index{Betriebsmodus!Cipher Block Chaining Mode (CBC-Modus)}\index{Cipher Block Chaining Mode (CBC-Modus)}}
\newcommand{\indexCTR}{\index{Betriebsmodus!Counter Mode (CTR-Modus)}\index{Counter Mode (CTR-Modus)}}

%Blockchiffren
\newcommand{\indexFeistel}{\index{Blockchiffre!Feistel networks}}
\newcommand{\indexDES}{\index{Blockchiffre!Data Encryption Standard (DES)}}
\newcommand{\indexTwoDES}{\index{Blockchiffre!2DES}}
\newcommand{\indexThreeDES}{\index{Blockchiffre!Triple Data Encryption Standard (3DES)}}
\newcommand{\indexAES}{\index{Blockchiffre!Advanced Encryption Standard (AES)}}
\newcommand{\indexConfusion}{\index{Confusion}}
\newcommand{\indexDiffusion}{\index{Diffusion}}
\newcommand{\indexSBOX}{\index{Substitution-Box (S-Box)}}

\newcommand{\indexIV}{\index{Initialisierungsvektor (IV)}}


\newcommand{\indexSecParam}{\index{Sicherheitsparameter}}

\newcommand{\indexNegl}{\index{Vernachlässigbarkeit}}

\newcommand{\indexINDCPA}{\index{Indistinguishability under chosen-plaintext attacks (IND-CPA)}}
\newcommand{\indexINDCCA}{\index{Indistinguishability under chosen-ciphertext attacks (IND-CCA)}}

\newcommand{\indexKeyExchange}{\index{Schlüsselaustausch}}

\newcommand{\indexOracle}{\index{Orakel}}

%Angriffe
\newcommand{\indexAttack}{\index{Angriff}}
\newcommand{\indexBruteForce}{\index{Angriff!Brute-Force / Exhaustive Search}}
\newcommand{\indexMeetInTheMiddle}{\index{Angriff!Meet-in-the-Middle}}
\newcommand{\indexManInTheMiddle}{\index{Angriff!Man-in-the-Middle}}
\newcommand{\indexDOS}{\index{Angriff!Denial of Service (DOS)}}
\newcommand{\indexLinCrypt}{\index{Lineare Kryptoanalyse}\index{Angriff!Lineare Kryptoanalyse}\index{Kryptoanalyse!Lineare Kryptoanalyse}}
\newcommand{\indexDiffCrypt}{\index{Differentielle Kryptoanalyse}\index{Angriff!Differentielle Kryptoanalyse}\index{Kryptoanalyse!Differentielle Kryptoanalyse}}
\newcommand{\indexBirthDayAttack}{\index{Angriff!Birthday-Angriff}}

\newcommand{\indexCryptAnalysis}{\index{Kryptoanalyse}}

%Angreifer
\newcommand{\indexAdv}{Angreifer}
\newcommand{\indexPassiveAdv}{\index{Angreifer!Passiv}}
\newcommand{\indexActiveAdv}{\index{Angreifer!Aktiv}}
\newcommand{\indexEfficientAdv}{\index{Angreifer!Effizient}}
\newcommand{\indexPPTAdv}{\index{Angreifer!Probabilistic Polynomial Time (PPT)}}

%Hashfunktionen
\newcommand{\indexHashFunction}{\index{Hashfunktion|see {Kryptographische Hashfunktion}}}
\newcommand{\indexCryptHashFunction}{\index{Kryptographische Hashfunktion}}
\newcommand{\indexCollisionResistance}{\index{Kryptographische Hashfunktion!Kollisionsresistenz}\index{Kollisionsresistenz}}
\newcommand{\indexPreImageResistance}{\index{Kryptographische Hashfunktion!Einwegeigenschaft}\index{Einwegeigenschaft}}
\newcommand{\indexTargetCollisionResistance}{Kryptographische Hashfunktion!Target Collision Resistance}\index{Target Collision Resistance}

\newcommand{\indexMDTransformation}{\index{Merkle-Damgård-Transformation}}

\newcommand{\indexMDFive}{\index{MD-5}}

%SHA
\newcommand{\indexSHA}{\index{Secure Hash Algorithm (SHA)}}
\newcommand{\indexSHAOne}{Secure Hash Algorithm (SHA)!SHA-1}
\newcommand{\indexSHATwo}{Secure Hash Algorithm (SHA)!SHA-2}
\newcommand{\indexSHAThree}{Secure Hash Algorithm (SHA)!SHA-3}

\newcommand{\indexEncryptionSymm}{\index{Verschlüsselung!symmetrisch}}
\newcommand{\indexEncryptionAsymm}{\index{Verschlüsselung!asymmetrisch}}

%RSA
\newcommand{\indexRSA}{\index{RSA}}
\newcommand{\indexRSATextBook}{\index{RSA!Textbook}}
\newcommand{\indexRSAHomomorphie}{\index{RSA!Homomorphie}}
\newcommand{\indexRSAOAEP}{\index{RSA!Optimal asymmetric encryption padding (OAEP)}}
\newcommand{\indexRSAPSS}{\index{RSA!Probabilistic Signature Scheme (PSS)}}
\newcommand{\indexEEA}{\index{Erweiterter Euklidischer Algorithmus (EEA)}}
\newcommand{\indexEulerPhiFunction}{\index{Eulersche Phi-Funktion}}
\newcommand{\indexFermatLittleTheorem}{\index{Fermat!Kleiner Satz}}
\newcommand{\indexChineseRemainderTheorem}{\index{Chinesischer Restsatz}}
\newcommand{\indexRandomOracleModel}{\index{Random Oracle Model}}

%ElGamal
\newcommand{\indexElGamal}{\index{ElGamal}}
\newcommand{\indexElGamalHomomorphie}{\index{ElGamal!Homomorphie}}
\newcommand{\indexDLOGProblem}{\index{DLOG-Problem}}
\newcommand{\indexDecisionalDiffieHellman}{\index{Decisional Diffie-Hellman-Annahme (DDH-Annahme)}}
\newcommand{\indexMessageTransformation}{\index{ElGamal!Nachrichtenumwandlung}}
\newcommand{\indexHashElGamal}{\index{Hash-ElGamal}}

%Authentifikation
\newcommand{\indexMessageAuthSymm}{\index{Nachrichtenauthentifikation!symmetrisch}}
\newcommand{\indexMessageAuthAsymm}{\index{Nachrichtenauthentifikation!asymmetrisch}}
\newcommand{\indexMAC}{\index{Message Authentication Code (MAC)}}
\newcommand{\indexHMAC}{\index{Keyed-Hash Message Authentication Code (HMAC)}}
\newcommand{\indexEUFCMA}{\index{Existential unforgeability under adaptive chosen message attacks (EUF-CMA)}}

\newcommand{\indexSig}{\index{Signatur}}

\newcommand{\indexPRF}{\index{Pseudorandomisierte Funktion (PRF)}}
\newcommand{\indexHashSign}{\index{Hash-then-Sign}}

\newcommand{\indexDSA}{\index{Digital Signature Algorithm (DSA)}}

\newcommand{\indexDigitalCert}{\index{Digitales Zertifikat}}
\newcommand{\indexCertAuthority}{\index{Certificate Authority (CA)}}
\newcommand{\indexXFiveZeroNine}{\index{Digitales Zertifikat!X.509}\index{X.509}}

\newcommand{\indexEncryptionHybrid}{\index{Verschlüsselung!hybrid}}
\newcommand{\indexKeyInfrastructure}{\index{Key-Infrastruktur}}
\newcommand{\indexSecretKeyInfrastructure}{\index{Key-Infrastruktur!Secret-Key-Infrastruktur}\index{Secret-Key-Infrastruktur}}
\newcommand{\indexPublicKeyInfrastructure}{\index{Key-Infrastruktur!Public-Key-Infrastruktur}\index{Public-Key-Infrastruktur}}
\newcommand{\indexKerberos}{\index{Kerberos}}
\newcommand{\indexPublicKeyTransport}{\index{Public-Key Transport}}
\newcommand{\indexDiffieHellmanKeyExchange}{\index{Diffie-Hellman-Schlüsselaustausch}}
\newcommand{\indexComputationalDiffieHellmanAssumption}{\index{Decisional Diffie-Hellman-Annahme (CDH-Annahme)}}
\newcommand{\indexComputationalDiffieHellmanProblem}{\index{Decisional Diffie-Hellman-Problem (CDH-Problem)}}
\newcommand{\indexTLS}{\index{Transport Layer Security (TLS)}}
\newcommand{\indexTLSHandshake}{\index{Transport Layer Security (TLS)!TLS-Handshake}}
\newcommand{\indexTLSPreMasterSecret}{\index{Transport Layer Security (TLS)!PreMaster Secret(PMS)}}
\newcommand{\indexTLSMasterKey}{\index{Transport Layer Security (TLS)!Master Key (MS)}}
\newcommand{\indexTransportSchicht}{\index{Transportschicht}}
\newcommand{\indexIPsec}{\index{Internet Protocol Security (IPsec)}}
\newcommand{\indexPAKE}{\index{Password Authentication Key Exchange (PAKE)}}

\newcommand{\indexPKIdentificationProtocoll}{\index{Public-Key-(PK-)Identifikationsprotokoll}}
\newcommand{\indexProver}{\index{Public-Key-(PK-)Identifikationsprotokoll!Prover}}
\newcommand{\indexVerifier}{\index{Public-Key-(PK-)Identifikationsprotokoll!Verifier}}
\newcommand{\indexChallengeResponce}{\index{Challenge-Response-Verfahren | see {Public-Key-(PK-)Identifikationsprotokoll}}}

\def\dh{d.\,h.\ }

%%% Umgebungen

\theoremstyle{plain}

\newtheorem{theorem}{Theorem}[chapter]
\newtheorem*{beweis}{Beweis}
\newtheorem*{beweisidee}{Beweisidee}
\newtheorem{beispiel}[theorem]{Beispiel}

\theoremstyle{definition}

\newtheorem{definition}[theorem]{Definition}